\chapter{标注小技巧}\label{ch:crossref}
\section{文档中的交叉引用}\label{sec:intro}
\LaTeX 中有许多{``}东西{''}是可以被标记和引用的,一般来说包括公式,图,表,章节。他们标记之后的引用是会自动编排的。而其他环境的引用取决于不同\LaTeX 的模板有没有给出定义。我们主要探究Elegantbook 这个模板里面一些可标记事物的交叉引用方法。
\begin{center}
\captionof{table}{可交叉引用的事物}
\label{tab:crossref}
    \begin{tabular}{|c|c|c|}
    \hline
    可以引用吗?  & 通常情况下 & Elegantbook 中 \\ \hline
    章\Verb"\chapter{}" & \checkmark& \checkmark \\ \hline
    节\Verb"\section{}" & \checkmark& \checkmark \\ \hline
    公式 & \checkmark& \checkmark \\ \hline
    图 & \checkmark& \checkmark \\ \hline
    表 & \checkmark& \checkmark \\ \hline
    定理\Verb"theorem" & -& \checkmark \\ \hline
    例题\Verb"example" & -& \checkmark \\ \hline
    证明\Verb"proof" & -& \ding{55} \\ \hline
    \end{tabular}
\end{center}
\subsection{一些实例}
我们在第\ref{ch:environment}章中主要讲了环境的使用,这一章讲如何进行交叉引用以及标注。本章的第\ref{sec:intro}节是一个简单的引入。其中图标可以通过\Verb"\ref{tab:crossref}"表\ref{tab:crossref}来引用,也可以通过\Verb"\tabref{tab:crossref}"\tabref{tab:crossref}来引用。

同理\Verb"\ref{fig:Ragdoll}"\ref{fig:Ragdoll}可以引用图片,也可以用\Verb"\figref{fig:Ragdoll}"\figref{fig:Ragdoll}来引用。
\begin{equation}\label{eq:TDSE}
    \hat{H} \Psi = \mathrm{i} \hbar \frac{\partial}{\partial t} \Psi.
\end{equation}
式\ref{eq:TDSE}被称为含时薛定谔方程。
\begin{equation}\label{eq:TISE}
    \hat{H} \Psi = E \Psi.
\end{equation}
式\eqref{eq:TISE}被称为不含时薛定谔方程。
\subsection{解题过程中添加交叉引用}
主要是使用符号注解,能够简易追溯公式中等号或者不等号的来源根据。下面来看一些例子
\begin{itemize}[wide, labelwidth=!,labelindent=0pt,leftmargin=*]
    \item 下括号或上括号
    \begin{equation*}
        \bm{0} = (\alpha_1 + \beta_1) \bm{x}_1 + \underbrace{\alpha_2 \bm{x}_2 + \cdots + \alpha_n \bm{x}_n}_{\text{互不相同}} \overbrace{- \beta_2 \bm{y}_2 - \cdots - \beta_m \bm{y}_m}^{\text{互不相同}}
    \end{equation*}
    \item 上下注解
    \begin{theorem}\label{theo:universal_compact}
    $(X,\mathcal{T})$是一个紧空间,当且仅当在$(X,\mathcal{T})$中所有的万有网都收敛。
    \end{theorem}
    \begin{equation*}
        (K,\mathcal{T}_K)\text{ 是紧空间}~~ \overset{\ref{theo:universal_compact}}{\Longleftrightarrow} ~~\forall K \text{中万有网} \{ x_\alpha \}_{\alpha \in I} \text{ 都满足 } x_\alpha \stackrel{\mathcal{T}_K}{\to} x \in K
    \end{equation*}
    \begin{equation*}
        (K,\mathcal{T}_K)\text{ 是紧空间}~~ \underset{\ref{theo:universal_compact}}{\Longleftrightarrow} ~~\forall K \text{中万有网} \{ x_\alpha \}_{\alpha \in I} \text{ 都满足 } x_\alpha \stackrel{\mathcal{T}_K}{\to} x \in K
    \end{equation*}
    \item \Verb"\usepackage{extarrows}"\newline
    由于$f$是在$(M, \mathbb{C})$上的线性映射,能够推出
    \begin{align}
        \Re f(\mathrm{i} x) & = - \Im f(x) \label{eq:RetoIm} \\
        \Im f(\mathrm{i} x) & = \Re f(x)
    \end{align}
    $\forall x \in M$,
    \begin{align*}
        \Lambda(x) = & g(x) - \mathrm{i} g(\mathrm{i} x) \\
        = & \Re f(x) - \mathrm{i} \Re f(\mathrm{i} x) \\
        \overset{\text{等式}\ref{eq:RetoIm}}{ = } &  \Re f(x) + \mathrm{i} \Im f(x) = f(x) \\
        \xlongequal{\text{等式}\ref{eq:RetoIm}} &  \Re f(x) + \mathrm{i} \Im f(x)
    \end{align*}
    \item \Verb"\usepackage{chemformula}"\newline
    用专门的化学包来写化学方程式会更加方便!
    \begin{center}
        \ch{2 H_2 O_2 ->[$\triangle$] 2 H_2 O + O_2 $\uparrow$}
    \end{center}
\end{itemize}
\begin{remark}
详情参照extarrows 的\href{https://ftp.fau.de/ctan/macros/latex/contrib/extarrows/extarrows-test.pdf}{ example of use}
\end{remark}
\begin{remark}
详情参照chemformular 的\href{https://ftp.mpi-inf.mpg.de/pub/tex/mirror/ftp.dante.de/pub/tex/macros/latex/contrib/chemformula/chemformula-manual.pdf}{ Documentation}
\end{remark}
\begin{remark}
更多符号的写法参照\href{https://oeis.org/wiki/List_of_LaTeX_mathematical_symbols}{ OEIS页面}
\end{remark}
\section{超链接}
用法:
\begin{verbatim}
    \href{链接}{文本}
\end{verbatim}
    
Ayumu 的\href{https://www.bilibili.com/video/BV1ve4y1m7hP/?spm_id_from=333.1007.top_right_bar_window_history.content.click&vd_source=4530026c0834d011205a1c8aae339ab9}{第二积分中值定理}是我最喜欢的一集!

\begin{remark}
每集都是最喜欢的一集是吧?!
\end{remark}
\section{引用}
\subsection{编译详情}
在Overleaf中,为了方便各位用户的使用,编译过程中的一些细节全部被整合到了\Verb"Compile"这个按钮里面。实际上,一个完整的\LaTeX 中文文档需要经过下面四部编译步骤:
\begin{verbatim}
    XeLaTeX -> bibTeX -> XeLaTeX -> XeLaTeX
\end{verbatim}    
或者
\begin{verbatim}
    XeLaTeX -> biber -> XeLaTeX -> XeLaTeX
\end{verbatim}    
\subsubsection{具体用法}
根据Elegantbook模板的版本不同,需要检查是否存在下列词条
\begin{itemize}
    \item 
    \begin{verbatim}
        \addbibresource[location=local]{reference.bib}
    \end{verbatim}
    \item 
    \begin{verbatim}
        \RequirePackage[
        backend=biber,
        citestyle=\ELEGANT@citestyle,
        bibstyle=\ELEGANT@bibstyle
        ]{biblatex}
    \end{verbatim}
\end{itemize}
若上述词条或命令都能找到的话,则在main.tex 同一个目录下创建reference.bib,然后将论文的引用复制进reference.bib 中。最后使用\Verb"\cite{}"命令进行引用。

\begin{verbatim}
     inkscape.com -D --export-filename=FILENAME.pdf --export-latex FILENAME.svg
\end{verbatim}


\begin{center}
    \def\svgwidth{0.7\columnwidth}
    \input{figure/pointwise.pdf_tex}
    \captionof{figure}{单调有上界函数必有极限。}
    \label{fig:pointwise}
\end{center}



\chapter{LaTeX基础用法}

%——————————————————————————————————%

\section{章节设置}


在LaTeX中,我们可以分章、节、小节、小小节等等。下面,我们看一个例子。

\subsection{LaTeX章节的优势}

在LaTeX中,章节是自动编号的,我们只需要输入反斜杠chapter, section, subsection, subsubsection,系统就会帮我们自动分节。

\subsection{分章节的方法}

一本书/讲义的Chapter数不会太多,一般是3-10章。每一章的内容基本独立,但好的讲义可以做到承前启后,章节之间彼此有联系。

一般来说,chapter和section就足够使用了。假如一个section的内容太多,我们才会考虑加入subsection,甚至subsubsection。

\subsection{换页的方法}

假如我们想要换一页,一种简单的方法就是输入反斜杠newpage的指令。要注意,一般来说我们只会在chapter间换页。为了举例,我们给大家看一下换页的效果。

\newpage



%——————————————————————————————————%

\section{行内公式与行间公式}

假如我们想要在一段话中插入数学公式,我们就需要用行内公式的用法,用一对dollar号围住公式就可以了。比如我们可以令$f(x)=x^2+3x+1$。打完公式以后,可以继续在同一段中写作。

有的时候,为了强调、为了美观,或者为了容易辨认,我们会使用行间公式。行间公式就是单独成一行,居中显示,默认用展示模式(display模式)。我们可以用两对dollar号围住公式,也可以用反斜杠加左右中括号的形式(我偏爱后一种用法)。比如我们令

\[ g(x)=\frac{ax^2+bx+c}{dx^2+ex+f}     \]

在这里,反斜杠frac加上两组花括号表示分式的形式。注意行间公式的分式形式是很美观的,但是行内公式的分式性质可能过于紧凑了。再次定义$g(x)=\frac{ax^2+bx+c}{dx^2+ex+f}$。很显然,行内公式不适合显示分式的公式。如果我们一定要在行内公式中输入分式,我们不妨直接用斜杠的写法。第三次定义$g(x)=\z(ax^2+bx+c\y)/\z(dx^2+ex+f\y)$。

大家看到了奇怪的反斜杠z和反斜杠y的记号。这是什么意思呢?这里,我们使用了newcommand的用法,这个记号并不是Overleaf或者LaTeX自带的,而是我们自己加上的。z和y分别指代了中文的“左”和“右”。本来的符号是反斜杠left和反斜杠right。用一对反斜杠的左、右框柱的一整个部分会被LaTeX识别为一个整体。当你要加括号的时候,LaTeX会自动调整括号的大小。

我们来看几个例子。

\[  g(x)^2=(\frac{ax^2+bx+c}{dx^2+ex+f})^2    \]

上式是很丑陋的。这是因为LaTeX并没有识别括号指代的范围。

\[ g(x)^2=\z(\frac{ax^2+bx+c}{dx^2+ex+f}\y)^2     \]

上式就很美观了。

类似地,我们也有其它用法。

\[  h(x)=\z| \frac{\sqrt{x}}{x^2+1} \y|    \]

在这里,反斜杠sqrt(square root)表示平方根。一个自然的问题是我们如何表示立方根或者$n$次方根。(在这里,$n$也用行内公式,这是因为我们的规范:任何数学符号、公式都要用公式的形式,不能用字母的形式)。

答案是用反斜杠sqrt中括号再加花括号的形式。

\[  k(x)=\z| \frac{\sqrt[5]{x^3+4}}{x^2+1} \y|    \]

再举几个例子。为了表并列,我们用枚举的用法,即反斜杠begin enumerate,每一个小点都用反斜杠item来表示。

\begin{enumerate}
    \item 
    \[  A=\z[a^2,a^2+1\y)    \]
    
    \item 
    \[ B=\z[\frac{a^2}{\sqrt[3]{b}},\frac{a^2+1}{\sqrt[3]{b}}\y) \]
    
    \item
    \[ C=\z\{x\in\R: \frac{x^2}{x^2+1}>1 \y\}    \]
\end{enumerate}

在最后一个例子中,我们注意到当我们要在公式中打出花括号的时候,需要用反斜杠花括号。这是因为一般的花括号是表整体,用花括号围住的部分指代了一个整体。为了区别于这个用法,我们只能用反斜杠z以及反斜杠y再加上反斜杠花括号来表示。

有时,我们并没有成对的需要用反斜杠z、y的符号,例如在数论中我们需要写整除,但是这一竖杠只有一个,并且需要适当地变大。我们可以用反斜杠big甚至反斜杠bigg的用法,但是我们也可以使用反斜杠z和反斜杠y,方法就是在不需要符号的地方加上一个点。

我们来看一个例子。

\[  n \z| \frac{a^p-a}{p} \y.        \]

在这里,右斜杠没有出现,但是整除号还是适应性地变大了。我们并没有违背反斜杠z和y应当成对出现的原则,我们用一个点来表示在这一边的符号是空缺的。

最后一个例子是用点乘定义的内积。

\[  \z< v,w\y>=v\cdot w        \]

同样地,这里的尖括号可以适应性地变大。

\[  \z< \frac{v}{c},\frac{w}{c}\y>=\frac{\z<v,w\y>}{c}        \]

%——————————————————————————————————%

\section{上标与下标}

我们总是需要较长的上标与下标。假如理解不清,就总会在需要用到的时候出现问题。因此,我们单独用一节来讲解上标与下标的使用方法。

我们或许最先用到的上标用法是一个较长的幂次。

\[  f(x)=x^{2y+1}    \]

注意,假如幂次的符号后没有加花括号,那么系统默认是只将幂次的符号后的第一个字符提到幂次的位置,而其他符号在默认的高度。我们通过下面的$g(x)$来和$f(x)$做对比。

\[  g(x)=x^2y+1    \]

为了效率,我们用以下的方法。
\begin{enumerate}
    \item 假如幂次上只有一个字符,我们不加花括号,例如$x^5+4x^3y^2+z^6+1$。
    \item 假如幂次上有超过一个字符,我们必须加花括号,例如$x^{a^2}+y^{bc}+e^{ab}$。
\end{enumerate}

假如我们要用多重上标,也必须用花括号的形式,例如上面的$x^{a^2}$。假如没有这个花括号,系统会报错。因此,我们用上面的原则,也可以避免这样的错误。

下标是类似的。我们也举几个简单的例子。比如令$x_1,\cdots,x_n\in\R$。或者等价地,令$\z\{x_i\y\}_{i=1,\cdots,n}\subset \R$。在这里,反斜杠cdots表示高度居中的三个点,subset表示子集,就是包含于的意思。

下标的时候,同样是一个字符的时候可以省略花括号,超过一个字符就必须使用花括号。嵌套的下标一定要用花括号。比如说数列$\z\{a_n\y\}_{n\in\N}\subset \R$的子列

\[  \z\{a_{k_n}\y\}_{n\in\N}\subset \R    \]

事实上,我们在很多场合下都可以使用上下标。例如表示一个有限的集合,我们可以同时使用上下标。

\[ A=\z\{x_i\y\}_{i=1}^{n}     \]

更常见的上下标是在大sigma记号,大pi记号下的,分别读作sum和product,表示求和与乘积。我们来看几个例子。

\[  \sum_{i=1}^ni=1+2+\cdots+n=\frac{n(n+1)}{2}    \]

\[ \prod_{i=1}^ni=n!     \]

当然,我们要注意,当我们在行内公式使用sum和prod的时候,公式也会变得丑陋起来。例如$\sum_{i=1}^ni=1+2+\cdots+n=\frac{n(n+1)}{2}$和$\prod_{i=1}^ni=n!$。

因此,我们在使用大的符号的时候,就应当大气地使用行间公式。

假如我们要对一个整体取幂次,也是用一样的方法。

\[  f(x)=\z( \frac{e^{e^{x_1}}}{x_{k_1}} \y)^{3y^{y^{y+1}}}    \]

总之就是只要学会了基础的用法,就可以不断迭代,玩出花来。重要的还是要打好基础。

用英文写作的时候,我们有时想要说第$n$个傅里叶系数。中文语境下说第$n$个就可以了,但是英文语境下要说 the $n^{\text{th}}$ Fourier coefficient. 在这里,在数学公式(无论是行内公式或行间公式)中输入反斜杠text,就是表示花括号内的内容是以文本形式输出(而非数学公式的形式)。

假如学到稍微困难一些的课,你可能知道直和的符号,这是用反斜杠oplus打出来的。例如一个有限维向量空间的直和分解。当我们在行间公式中使用直和记号的时候,我们应该使用反斜杠bigoplus。

\[  V=\bigoplus_{i=1}^n V_i        \]

%——————————————————————————————————%

\section{Elegantbook模板中的定理环境}

现在,既然我们已经使用了Elegantbook的模板,我们当然应该使用其中的定理环境。

我们分别来演示定义、引理、证明、命题、例题、练习、批注的用法。

\begin{definition}[有理数]
我们称$p/q$是一个有理数,当且仅当
\begin{enumerate}
    \item $p,q$是两个整数。
    \item $q$不等于0。
    \item $p,q$的最大公因数是1,或$p,q$互素。
\end{enumerate}

我们称由全体有理数构成的集合为有理数集,记作$\Q$。
\end{definition}

\begin{lemma}[重要引理]
假设$x\in \R$,则$x^2\ge 0$。
\end{lemma}
\begin{proof}
分类讨论。
\begin{enumerate}
    \item 假设$x\ge 0$,则$x^2=x\cdot x\ge 0$。
    \item 假设$x<0$,则$-x>0$,我们有$x^2=(-x)^2\ge 0$。
\end{enumerate}

此即得证。
\end{proof}

\begin{proposition}[重要命题]
假设$x,y\in\R$,则$x^2+y^2\ge 2xy$。
\end{proposition}
\begin{proof}
令$x,y\in\R$。利用上面的引理,我们知道$(x-y)^2=x^2-2xy+y^2\ge 0$。因此显然有
\[  x^2+y^2\ge 2xy    \]

此即得证。
\end{proof}

\begin{remark}
通过这样的写法,我们可以让讲义变得更加美观、可读,让学生更愿意读下去,让老师更有成就感。
\end{remark}

%——————————————————————————————————%

\section{公式对齐}

有时我们需要用到递等式,或者一般地,只是想要将一些符号对齐。

我们一般用begin align*的形式。加星号是为了不给公式编号。我们用\&来表示需要紧跟着的对齐的符号。我们用双反斜杠来表示换行。结合这么几点,我们就学会了公式对齐的用法。
\begin{align*}
(a+b)(c+d) &=
a(c+d)+b(c+d)\\
&=ac+ad+bc+bd\\
&=ac+bc+ad+bd
\end{align*}

当然,我们也可以用连续的不等式。
\begin{align*}
|a+b+c+d|
&\le |a+b|+|c+d|  \\
&\le |a|+|b|+|c|+|d|
\end{align*}

或者命题的连续等价。
\begin{align*}
&(x,y)\in (A\times B)\cap(C\times D)\\
\iff &(x,y)\in A\times B, (x,y)\in C\times D\\
\iff &x\in A,y\in B,x\in C,y\in D\\
\iff &x\in A\cap C,y\in B\cap D\\
\iff &(x,y)\in (A\cap C)\times (B\cap D)
\end{align*}

%——————————————————————————————————%

\section{特殊符号}

下面,我们来看一些例子。

\begin{enumerate}
    \item $\sin^2+\cos^2=1$。
    \item $\log (\exp x)  =x$。
    \item $\gcd(a,b)\lcm(a,b)=ab$。
\end{enumerate}

\subsection{实数}
下面,我们看实数中的一些例子。
\begin{enumerate}
    \item 小于:$<$。
    \item 大于:$>$。
    \item 小于等于:$\le$或者$\leqslant$。
    \item 大于等于:$\ge$或者$\geqslant$。
    \item 不等于:$\neq$。
\end{enumerate}

\subsection{数理逻辑}
下面,我们看数理逻辑中的一些例子。
\begin{enumerate}
    \item 与运算:$\land$。
    \item 或运算:$\lor$。
    \item 非运算:$\lnot$。
    \item 蕴含:$\to$。
    \item 等价:$\leftrightarrow$或者$\iff$。
    \item 全称量词:$\forall$。
    \item 存在量词:$\exists$。
\end{enumerate}

\subsection{集合论}
下面,我们来看集合论中的一些例子。
\begin{enumerate}
    \item 属于:$\in$。
    \item 包含元素:$\ni$。
    \item 子集/包含于:$\subset$。
    \item 包含:$\supset$。
    \item 真包含于:$\subsetneq$。
    \item 真包含:$\supsetneq$。
    \item 并:$\cup$或者$\bigcup$。
    \item 交:$\cap$或者$\bigcap$。
    \item 差:$-$或者$\setminus$。
    \item 补:$\cdot^C$。
    \item 笛卡尔积:$\times$或者$\prod$。
    \item 无交并:$\sqcup$或者$\bigsqcup$。
\end{enumerate}

\subsection{关系与映射}
下面,我们介绍关系与映射中的一些记号。
\begin{enumerate}
    \item 同胚/同构:$\simeq$。
    \item 等价:$\sim$。
    \item 映射:$f:A\to B$。
    \item 复合:$\circ$。
\end{enumerate}


\subsection{微积分/数学分析}
下面是微积分/数学分析中的一些例子。

\[  f'(x)=\lim_{h\to 0}\frac{f(x+h)-f(x)}{h} \]

\[  \int_a^bf(x)dx=F(b)-F(a) \]

\[ \int_a^bf(x)dx=\lim_{n\to \infty }\sum_{i=1}^n f\z(x_i^*\y)\z(x_i-x_{i-1}\y)     \]

\[  \limsup_{n\to \infty}a_n=\inf_{n\ge 1}\sup_{k\ge n}a_k   \]

\[  \liminf_{n\to \infty}a_n=\sup_{n\ge 1}\inf_{k\ge n}a_k   \]

\subsection{矩阵}
下面,我们来看矩阵的写法。
\[  A=\begin{pmatrix}
1&2&3\\
4&5&6\\
7&8&9\\
\end{pmatrix}\]

$A$的行列式记为
\[ \det A=\begin{vmatrix}
1&2&3\\
4&5&6\\
7&8&9\\
\end{vmatrix} \]

\subsection{bar和tilde}
下面,我们来看bar和tilde的写法。

在复分析中,我们对元素取共轭就是加一个bar。很多时候,bar太短,我们用overline来代替。
\[  \overline{z+w}=\overline{z}+\overline{w} \]

在对函数$f$做修改后,我们有时用$\tilde{f}$来表示修改后的函数。它们很相似,但又有不同。我们如果用$f'$,就容易与导数相混淆。为了避免歧义,我们有时用tilde来表示。







